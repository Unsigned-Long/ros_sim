\documentclass[12pt, twocolumn]{article}

% 引入相关的包
\usepackage{amsmath, listings, fontspec, geometry, graphicx, ctex, color, subfigure}

% 设定页面的尺寸和比例
\geometry{left = 1.5cm, right = 1.5cm, top = 1.5cm, bottom = 1.5cm}

% 设定两栏之间的间距
\setlength\columnsep{1cm}

% 设定字体,为代码的插入作准备
\newfontfamily\ubuntu{Ubuntu Mono}

% 头部信息
\title{\normf{这是标题}}
\author{\normf{陈烁龙}}
\date{\normf{\today}}

% 代码块的风格设定
\lstset{
	language=C++,
	basicstyle=\scriptsize\ubuntu,
	keywordstyle=\textbf,
	stringstyle=\itshape,
	commentstyle=\itshape,
	numberstyle=\scriptsize\ubuntu,
	showstringspaces=false,
	numbers=left,
	numbersep=8pt,
	tabsize=2,
	frame=single,
	framerule=1pt,
	columns=fullflexible,
	breaklines,
	frame=shadowbox, 
	backgroundcolor=\color[rgb]{0.97,0.97,0.97}
}

% 字体族的定义
\newcommand{\normf}{\kaishu}
\newcommand{\boldf}{\heiti}
\newcommand\keywords[1]{\boldf{关键词:} \normf #1}

\begin{document}
	
	% 插入头部信息
	\maketitle
	% 换页
	\thispagestyle{empty}
	\clearpage
	
	% 插入目录、图、表并换页
	\tableofcontents
	\listoffigures
	\listoftables
	\setcounter{page}{1}
	% 罗马字母形式的页码
	\pagenumbering{roman}
	\clearpage
	% 从该页开始计数
	\setcounter{page}{1}
	% 阿拉伯数字形式的页码
	\pagenumbering{arabic}
	
	
	\section{\normf{GAZEBO仿真}}
	\normf
	通过Blender软件绘制仿真场景,将其导入出为dae文件。而后将dae文件导入到gazebo,保存成world(是sdf文件)。当然,该场景可能相较于blender场景偏暗,可以通过直接调整dae文件中的:
\begin{lstlisting}[label=code2,caption={\normf 亮度调整}]
<emission>
	<color sid="emission">0 0 0 1</color>
</emission>
<diffuse>
	<color sid="emission">0 0 0 1</color>
</diffuse>
	\end{lstlisting}
	标签来实现。
	
	另外,场景可能没有纹理,这时打开保存的sdf文件,将里面的此代码删除即可:
\begin{lstlisting}[label=code2,caption={\normf 纹理缺失}]
<material>
	<script>
		<uri>file://media/materials/scripts/gazebo.material</uri>
		<name>Gazebo/Grey</name>
	</script>
</material>
\end{lstlisting}

	通过Blender,可以自动生成轨迹(基于控制点和贝塞尔曲线插值)。导入到gazebo时,可以借助actor实现:
\begin{lstlisting}[label=code2,caption={\normf 轨迹}]
<actor name="sim_robot">
    <link name="base_footprint">
    ...
    </link>
    <script>
    ...
    </script>
</actor>
\end{lstlisting}
link标签是由urdf或者xacro机器人文件转为sdf后的link节点。而script则是由blender生成的轨迹处理得到的。

\section{\normf{框架使用}}
注意到,update-model.sh脚本文件可以将xacro的文件转为urdf和sdf格式文件,用于产生actor标签的link节点内容。而可执行程序generate-actor可以将link节点和blender生成轨迹拼接,得到actor节点。
	
\end{document}

